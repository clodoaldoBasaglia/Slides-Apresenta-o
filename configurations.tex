\usepackage[utf8]{inputenc}
\usepackage[brazil]{babel}  % idioma
%\usepackage[T1]{fontenc}

\usepackage{utopia} %font utopia imported

\usepackage{framed}

%\usetheme{Madrid}
% \usecolortheme{goeagles}
%\usecolortheme{default}

%\usetheme{Frankfurt}
\usetheme{Madrid}
%\usecolortheme[rgb={0.23,0.27,0.29}]{structure}
\usecolortheme[rgb={0.21, 0.27, 0.31}]{structure}

%\usecolortheme{crane}
\setbeamercovered{transparent}

\usefonttheme[onlymath]{serif} % fonte modo matematico

\usepackage{graphics}
\usepackage{subfigure}
\usepackage{float}
\usepackage{multirow}
\usepackage{verbatim}
\usepackage{remreset}
\usepackage{comment} % end and begin comment
%\usepackage{dtklogos} % for \BibTeX
\usepackage{listings} % display code on slides; don't forget [fragile] option after \begin{frame}
\usepackage{color}
\usepackage{url}
\usepackage{tikz}
\usepackage[tikz]{bclogo}

\usepackage{ragged2e} % justifying.

\usepackage{lstlisting-llvm}

\usepackage{adjustbox} % ajustar codigo.

\usepackage[normalem]{ulem} % tachado.

\usepackage{xkeyval}
\usepackage{todonotes}
\presetkeys{todonotes}{inline}{}


%\usepackage{enumitem} % alterar a distancia dos itens dos itemize.

% Colocar boxes com descricao de figuras.
\usepackage{graphicx}
\newcommand{\putat}[3]{\begin{picture}(0,0)(0,0)\put(#1,#2){#3}\end{picture}}

% Opçõess de listing usados para o codigo fonte
% Ref: http://en.wikibooks.org/wiki/LaTeX/Packages/Listings
\lstset{ %
language=Java,                  % choose the language of the code
basicstyle=\footnotesize,       % the size of the fonts that are used for the code
%basicstyle=\ttfamily,
stringstyle=\ttfamily\color[rgb]{0.16,0.16,0.16},
numbers=left,                   % where to put the line-numbers
numberstyle=\footnotesize,      % the size of the fonts that are used for the line-numbers
stepnumber=1,                   % the step between two line-numbers. If it's 1 each line will be numbered
numbersep=2pt,                  % how far the line-numbers are from the code
showspaces=false,               % show spaces adding particular underscores
showstringspaces=false,         % underline spaces within strings
showtabs=true,					% show tabs within strings adding particular underscores
frame=single,	                % adds a frame around the code
framerule=0.6pt,
tabsize=2,	                	% sets default tabsize to 2 spaces
%keepspaces,					% set one line at code final.
extendedchars=true,
captionpos=b,                   % sets the caption-position to bottom
breaklines=true,                % sets automatic line breaking
breakatwhitespace=false,        % sets if automatic breaks should only happen at whitespace
aboveskip=5pt,
upquote=true,
columns=fixed,
escapeinside={\%*}{*)},         % if you want to add a comment within your code
backgroundcolor=\color[rgb]{1.0,1.0,1.0}, % choose the background color.
rulecolor=\color[rgb]{0.8,0.8,0.8},
xleftmargin=10pt,
xrightmargin=10pt,
framexleftmargin=10pt,
framexrightmargin=10pt
}

%%% RAG----------------------------------------------------------------%
%-----------------------Style Definitions------------------------------%
\definecolor{javared}{rgb}{0.6,0,0} % for strings
\definecolor{javagreen}{rgb}{0.25,0.5,0.35} % comments
\definecolor{javapurple}{rgb}{0.5,0,0.35} % keywords
\definecolor{javadocblue}{rgb}{0.25,0.35,0.75} % javadoc

\definecolor{DarkBlue}{rgb}{0,0,0.61}
\definecolor{DarkGreen}{rgb}{0,0.4,0}
\definecolor{DarkRed}{rgb}{0.51,0,0}
\definecolor{DarkBlue2}{rgb}{0.25,0.51,0}


% Numeros.
\lstdefinestyle{mynumbers}{
	numbers=left,
	stepnumber=1,
	numbersep=4pt,
	numberstyle=\tiny\color{black}
}
% Text Code.
\lstdefinestyle{mytextcode}{
	basicstyle=\footnotesize,
	tabsize=2,
	showspaces=false,
	showstringspaces=false,
	extendedchars=true,
	breaklines=true
}
% Frame.
\lstdefinestyle{myframe}{
	backgroundcolor=\color{white},
	frame=trbl
}
% C++ Style.
\lstdefinestyle{C++}{
	language=C++,
	style=mynumbers,
	style=mytextcode,
	style=myframe,
	keywordstyle=\color{black}\bfseries,
	stringstyle=\color{gray},
	commentstyle=\color[rgb]{0.08,0.08,0.08},
	morecomment=[s][\color{lightgray}]{/*}{*/},
	otherkeywords={dim3},
  	emph={ \_\_device\_\_, \_\_global\_\_, \_\_shared\_\_, \_\_host\_\_, \_\_constant\_\_},
	emphstyle=\color{DarkBlue}\bfseries,
	emph={[2] printf, scanf, \#include, \#define, \#pragma, \#typedef},
	emphstyle=[2]\color{DarkGreen},
}
% C Style.
\lstdefinestyle{C}{
	language=C,
	style=mynumbers,
	style=mytextcode,
	style=myframe,
	basicstyle=\ttfamily,
	keywordstyle=\color{javapurple}\bfseries,
  	stringstyle=\color{gray}\bfseries,
  	commentstyle=\color[rgb]{0.08,0.08,0.08},
  	morecomment=[s][\color{lightgray}]{/*}{*/},
  	otherkeywords={dim3, \#define, \#pragma, \#typedef},
  	emph={ \_\_device\_\_, \_\_global\_\_, \_\_shared\_\_, \_\_host\_\_, \_\_constant\_\_},
  	emphstyle=\color{DarkBlue}\bfseries,
  	emph={[2] printf, scanf, \#include},
  	emphstyle=[2]\color{DarkGreen},
  	emph={[3] omp, oac},
  	emphstyle=[3]\color{DarkRed},
  	emph={[4] single, parallel, runtime},
  	emphstyle=[4]\color{DarkBlue2},
	backgroundcolor={},
	identifierstyle=\color{black}	
}
% Bash Style.
\lstdefinestyle{bash}{
	language=bash,
	style=mynumbers,
	style=mytextcode,
	style=myframe,
	backgroundcolor={},
	frame=single,
	basicstyle=\scriptsize\ttfamily
}
% Python Style.
\lstdefinestyle{python}{
	language=python,
	style=mynumbers,
	style=mytextcode,
	style=myframe,
	backgroundcolor={}
}
% Java Style.
\lstdefinestyle{java}{
	language=java,
	style=mynumbers,
	style=mytextcode,
	style=myframe,
	backgroundcolor={},
	basicstyle=\ttfamily,
	keywordstyle=\color{javapurple}\bfseries,
	stringstyle=\color{javared},
	commentstyle=\color{javagreen},
	morecomment=[s][\color{javadocblue}]{/**}{*/}
}
% ASM Style.
\lstdefinestyle{asm}{
  %belowcaptionskip=1\baselineskip,
  %xleftmargin=\parindent,
  language=[x86masm]Assembler,
  style=mynumbers,
  style=mytextcode,
  style=myframe,
  backgroundcolor={},
  frame=single,
  basicstyle=\scriptsize\ttfamily,
  keywordstyle=\color{javapurple}\bfseries,
  stringstyle=\color{gray}\bfseries,
  commentstyle=\itshape\color{red!40!gray},
  identifierstyle=\color{black},
  otherkeywords={movl, leaq, movq, subq, jmp, jg, pushq, popq, addl, cmpl, movss, cmpb, idivl, cltd},
  emph={eax, ebx, ecx, edx, esi, edi, ebp, eip, esp, r8d, r9d, r10d, r11d, r12d, r13d, r14d, r15d},
  emphstyle=\color{DarkGreen}\bfseries,
  emph={[2]rax, rbx, rcx, rdx, rsi, rdi, rbp, rip, rsp, r8, r9, r10, r11, r12, r13, r14, r15},
  emphstyle=[2]\color{DarkRed}\bfseries  
}

% Fortran Style.
\lstdefinestyle{fortran}{
  language=[90]Fortran,
  style=mynumbers,
  style=mytextcode,
  style=myframe,
  backgroundcolor={},
  frame=single,
  basicstyle=\footnotesize,
  commentstyle=\itshape\color{purple!40!black},
  morecomment=[l]{!\ }% Comment only with space after !
}

% LLVM Style.
\lstdefinestyle{llvm}{
	language=llvm,
	%inputencoding=utf8,
	style=mynumbers,
	style=mytextcode,
	style=myframe,
	backgroundcolor={},
	frame=single,
	basicstyle=\scriptsize\ttfamily,
  tabsize=4,
  %rulecolor=,
  upquote=true,
% aboveskip={1.5\baselineskip},
  columns=fixed,
  prebreak = \raisebox{0ex}[0ex][0ex]{\ensuremath{\hookleftarrow}},
  showtabs=false,
	%basicstyle=\scriptsize\upshape\ttfamily,
  identifierstyle=\ttfamily,
  keywordstyle=\ttfamily\bfseries\color[rgb]{0,0,0},
  %commentstyle=\ttfamily\color[rgb]{0.133,0.545,0.133},
  commentstyle=\ttfamily\color[rgb]{0.08,0.08,0.08},
  %stringstyle=\ttfamily\color[rgb]{0.627,0.126,0.941}
  stringstyle=\ttfamily\color[rgb]{0.16,0.16,0.16},
  morecomment = [l]{;},
    morestring=[b]", 
    sensitive = true,
    classoffset=0,
    morekeywords={
      define, declare, global, constant,
      internal, external, private,
      linkonce, linkonce_odr, weak, weak_odr, appending,
      common, extern_weak,
      thread_local, dllimport, dllexport,
      hidden, protected, default,
      except, deplibs,
      volatile, fastcc, coldcc, cc, ccc,
      x86_stdcallcc, x86_fastcallcc,
      ptx_kernel, ptx_device,
      signext, zeroext, inreg, sret, nounwind, noreturn,
      nocapture, byval, nest, readnone, readonly, noalias, uwtable,
      inlinehint, noinline, alwaysinline, optsize, ssp, sspreq,
      noredzone, noimplicitfloat, naked, alignstack,
      module, asm, align, tail, to,
      addrspace, section, alias, sideeffect, c, gc,
      target, datalayout, triple,
      blockaddress
    },
    classoffset=1, keywordstyle=\color{purple},
    morekeywords={
      fadd, sub, fsub, mul, fmul,
      sdiv, udiv, fdiv, srem, urem, frem,
      and, or, xor,
      icmp, fcmp,
      eq, ne, ugt, uge, ult, ule, sgt, sge, slt, sle,
      oeq, ogt, oge, olt, ole, one, ord, ueq, ugt, uge,
      ult, ule, une, uno,
      nuw, nsw, exact, inbounds,
      phi, call, select, shl, lshr, ashr, va_arg,
      trunc, zext, sext,
      fptrunc, fpext, fptoui, fptosi, uitofp, sitofp,
      ptrtoint, inttoptr, bitcast,
      ret, br, indirectbr, switch, invoke, unwind, unreachable,
      malloc, alloca, free, load, store, getelementptr,
      extractelement, insertelement, shufflevector,
      extractvalue, insertvalue,
    },
    alsoletter={\%},
    keywordsprefix={\%},
}
%%% RAG----------------------------------------------------------------%

\renewcommand{\lstlistingname}{Código}

% \newcommand{\putat}[3]{\begin{picture}(0,0)(0,0)\put(#1,#2){#3}\end{picture}}

\setbeamertemplate{caption}[numbered]


% % % RAG -------------------------------------------------------------- %
% References.
\usepackage{natbib}

% % % RAG -------------------------------------------------------------- %
% Caixas
\usepackage[most]{tcolorbox}
\usepackage{lipsum}

\usepackage{algorithm,algorithmic}

\usepackage{multicol}
\usepackage{dot2texi}
\usepackage{tikz}
\usetikzlibrary{shapes,arrows}
\usepackage{graphviz}