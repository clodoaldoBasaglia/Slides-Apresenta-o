\documentclass{beamer}
\usepackage{minted}
\usepackage{outlines}
\include{configurations}
%\documentclass[12pt]{beamer}
%\documentclass[handout]{beamer}
% \setbeameroption{show notes}
% \setbeamertemplate{note page}[plain]
\usepackage{etex}
\usepackage{minted}
\usepackage{pgfpages}
% inclui o modo handout.
\include{seisporpagina}
\include{colors-defs}
\newcommand{\classnumber} {}
\newcommand{\classtitle} {Veículo autônomo terrestre de baixo custo para uso agrícola}
\newcommand{\classsubtitle} {}
\newcommand{\coursename} {}
\newcommand{\subjectname} {}
\newcommand{\idsubjectcourseinstitution} {Veículo autônomo terrestre de baixo custo}
\newcommand{\universityname}{Universidade Tecnológica Federal do Paraná (UTFPR)}
\newcommand{\departmentname}{Departamento de Computação (DACOM)}
\newcommand{\citystatecountry}{Campo Mourão, Paraná, Brasil}
\usepackage{appendixnumberbeamer}
\newtcolorbox{mybox}[3][]
{
	breakable,
	boxsep=1pt,
	left=2pt,
	right=2pt,
	top=1pt,
	bottom=1pt,
	colframe = #2!40!black,
	colback  = #2!10,
	% coltitle = #2!20!black,  
	title    = #3,
	#1,
}

%-----------------------------------------------------
%This block of code defines the information to appear in the
%Title page
\title[\idsubjectcourseinstitution] %optional
{\texttt{\classnumber} \classtitle}
\subtitle{\classsubtitle}
\normalsize
\author[Clodoaldo da Fonseca] % (optional)
{\\\\Clodoaldo A. Basaglia da Fonseca \\Prof. Dr. Radames Juliano Halmeman
	\texttt{}\\
 \and
 \inst{}\\
    \texttt{}
}

\normalsize
\institute[] % (optional)
{
 %\inst{1}%
  \universityname \\
  \departmentname \\
  \citystatecountry
  \and
  \large
  \textbf{\coursename}
  
  \vspace{-1.0cm}
}
\date[\today] % (optional)
{\vspace{-1.5cm} \texttt{\subjectname}}

% \logo{\includegraphics[height=0.7cm]{logos/logos.png}}
%\titlegraphic{\includegraphics[width=2cm]{logos/logos.png}\hspace*{4.75cm}~%
%	\includegraphics[width=2cm]{logos/logos.png}
%}

% \titlegraphic{\includegraphics[height=0.7cm]{logos/logos.png}\vspace*{-0.2cm}}

\setbeamertemplate{navigation symbols}{}

%End of title page configuration block
%-------------------------------------------------------

%-----------------------------------------------------
%The next block of commands puts the table of contents at the 
%beginning of each section and highlights the current section:

%\AtBeginSection[]
%{
%  \begin{frame}
%    \frametitle{Agenda}
%    \tableofcontents[currentsection]
%  \end{frame}
%}
%-----------------------------------------------------
\begin{document}

%The next statement creates the title page.
%\frame{\titlepage}
%\note{note}

{
	\setbeamertemplate{headline}{}
	%\usebackgroundtemplate{\includegraphics[trim=1.0cm 1.0cm 1.0cm 1.0cm, scale=0.47]{logos/header-utfpr.pdf}}
	\pgfdeclareimage[width=\paperwidth,height=\paperheight]{background}{logos/header-utfpr.pdf}
	\usebackgroundtemplate{\pgfuseimage{background}}

	% logo
	\vspace{0.5cm}
	\pgfdeclareimage[height=0.28cm, width=1.0cm]{logo}{logos/logo.png}
	\logo{\pgfuseimage{logo}}
	\begin{frame}
		\maketitle
	\end{frame}
}

% Outros slides.
\pgfdeclareimage[width=\paperwidth,height=\paperheight]{background}{logos/header-utfpr.pdf}
\usebackgroundtemplate{\pgfuseimage{background}}

% logo
\vspace{0.5cm}
\pgfdeclareimage[height=0.28cm, width=1.0cm]{logo}{logos/logo.png}
\logo{\pgfuseimage{logo}}

%---------------------------------------------------------------------------
%This block of code is for the table of contents after
%the title page




%---------------------------------------------------------------------------
%---------------------------------------------------------------------------
%This block of code is for the table of contents after
%the title page
\begin{frame}
\frametitle{Contextualização}
\begin{outline}
 \1 Mão de obra no campo está cada vez mais escassa
 \2 Segundo o IBGE, em 1950 cerca de 63,8\% da população brasileira era rural e no ano de 2010,  15,65\% da população brasileira é rural
 \1 Encarecimento da produção agrícola por desperdício de insumos
 \1 Necessidade de modernização da frota agrícola
\end{outline}
\end{frame}

%---------------------------------------------------------------------------
\section[]{Evolução}

\begin{frame}
\frametitle{Evolução}

\begin{figure}[!htb]
\centering
\includegraphics[scale=0.25]{imgs/arado1900.jpg}
%\caption{Arado de tração animal}
\label{Rotulo}
\end{figure}


\end{frame}
%---------------------------------------------------------------------------
%---------------------------------------------------------------------------
\section[]{Evolução}

\begin{frame}
\frametitle{Evolução}

\begin{figure}[!htb]
\centering
\includegraphics[scale=0.25]{imgs/tratoratual.jpg}
%\caption{Trator operado por um humano}
\label{Rotulo}
\end{figure}


\end{frame}
%---------------------------------------------------------------------------
%---------------------------------------------------------------------------
\section[]{Evolução}

\begin{frame}
\frametitle{Evolução}

\begin{figure}[!htb]
\centering
\includegraphics[scale=0.25]{imgs/autonomous.jpeg}
%\caption{Trator Autônomo da Case}
\label{Rotulo}
\end{figure}


\end{frame}
%---------------------------------------------------------------------------
%---------------------------------------------------------------------------
\section[Justificativa]{Justificativa}

\begin{frame}
\frametitle{Justificativa}
%\vspace{-20px}
\begin{itemize}
    \item Dado a escassez de mão de obra, afeta-se diretamente a capacidade de produção
    \item Aplicação de quantidades erradas insumos agrícolas podem aumentar os custos de produção e causar danos ambientais
    \item Automação pode aumentar a competitividade
\end{itemize}

\justifying
\Large
\end{frame}
%-------------------------------------------------------------------

%-----------------------------------------------------
\section[Objetivo]{Objetivo}

\begin{frame}
\frametitle{Objetivos}
%\vspace{-20px}
\begin{outline}
 \1 Desenvolver um veículo autônomo de baixo custo capaz de percorrer fileiras em plantações
 \2 Plataforma que possa acomodar diferentes funções
 \2 Algoritmo para percorrer as fileiras sem repetir passagens
 \2 Veículo elétrico com fonte de energia fotovoltaica
 \2 Veículo de baixo custo torna seu uso possível na agricultura familiar
\end{outline}

\justifying
\Large
\end{frame}
%---------------------------------------------------------------------------
%-----------------------------------------------------
\section[Embasamento]{Embasamento Teórico}

\begin{frame}
\frametitle{Embasamento Teórico}
%\vspace{-20px}
\begin{outline}
\1 BRACHT, Marcos P. Pettermann. Projeto de um veículo autônomo capaz de cobrir uma área poligonal sem passar mais de uma vez pela mesma região. 2015.
\1 Pawin Thanpattranon,Tofael Ahamed e Tomohiro Takigawa.Navigation of an Autonomous Tractor for a Row-Type Tree Plantation Using a Laser Range Finder—Development of a Point-to-Go Algorithm.2015

\end{outline}

\justifying
\Large
\end{frame}
%---------------------------------------------------------------------------
%-----------------------------------------------------
\section[Trabalhos]{Trabalhos Relacionados}

\begin{frame}
\frametitle{Trabalhos Relacionados}
%\vspace{-20px}
\begin{outline}
 \1 M Bietresato, G Carabin. A Tracked Mobile Robotic Lab for Monitoring the Plants Volume and Health. 
 \1 PEREIRA, Luiz Felipe. C.E.R.B.E.R.U.S.: Robo bombeiro. 2016
 \1 MARQUES, Leomar. Projeto Conceitual de Robô terrestre rádio Controlado de baixo custo para inspeção e transporte de materiais perigosos. 2016
\end{outline}

\justifying
\Large
\end{frame}
%---------------------------------------------------------------------------
%---------------------------------------------------------------------------
\section[]{Fim}

\begin{frame}
\frametitle{Dúvidas?}
\begin{center}
    Dúvidas?\\
    clodaoldofonseca92@gmail.com
\end{center}

\end{frame}
%---------------------------------------------------------------------------

\end{document}
